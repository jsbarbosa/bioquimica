\documentclass[12pt]{article}
\usepackage{geometry}
\usepackage[utf8]{inputenc}
\usepackage[spanish]{babel}
\usepackage[square,numbers, sort]{natbib}
\usepackage[colorlinks, citecolor = blue, linkcolor = blue]{hyperref}
\usepackage[dvipsnames]{xcolor}
\usepackage{enumitem}
\usepackage{sectsty}
\usepackage{multicol}
\usepackage{fancyvrb}

%\chapterfont{\color{blue}}  % sets colour of chapters
\sectionfont{\color{blue}}
\subsectionfont{\color{blue}}
\subsubsectionfont{\color{blue}}
\paragraphfont{\color{blue}}


\title{\color{blue} \scshape{GAPDH\\ enzima recombinante}}
\author{Juan Barbosa - 201325901}

\newcommand{\enzima}{\textbf{GAPDH}}

\begin{document}
	\maketitle
%	\dominitoc
	\begin{multicols}{2}
		\footnotesize
		\tableofcontents
	\end{multicols}
	\rule{15cm}{0.4pt}

	\section{C\'omo extraer el DNA}
		Dado que la \enzima{} tiene un origen humano, la muestra de DNA ser\'ia obtenida a partir de sangre humana. Con el objetivo de preservar la muestra, hasta el momento de la extracci\'on del material gen\'etico, a la sangre se le agregar\'ia EDTA como anticoagulante, y ser\'a conservada a 4 $^\circ$C \cite{m2011human, puregeneBook}.
		\subsection{M\'etodo de extracci\'on}
			Considerando una muestra de 300 $u$L de sangre, y usando el kit \textit{Gentra Puregene Blood Kit} para extraer el DNA, los pasos necesarios son los siguientes:
			\begin{enumerate}[label=\color{blue}\theenumi]
				\item Agregar 0.9 mL de la soluci\'on de lisis RBC (\textit{Red Blood Cells}) a un falcon
				\item Agregar 300 $\mu$L de la muestra de sangre al mismo falcon y mezclar 10 veces por inversi\'on
				\item Incubar la mezcla a 25 $^\circ$C por 1 minutos
				\item Centrifugar a 13000 x por 20 segundos para precipitar las c\'elulas blancas
				\item Retirar el sobrenadante dejando 10 $\mu$L en el falcon
				\item Agitar el falcon para resuspender el residuo s\'olido
				\item Agregar 300 $\mu$L de la soluci\'on de lisis celular, y agitar
				\item Agregar 1.5 $\mu$L de \textit{RNaseA Solution}, mezclar y encubar a 37 $^\circ$C por 15 minutos
				\item Agregar 0.1 mL de la soluci\'on para la precipitaci\'on de prote\'inas y agitar
				\item Centrifugar por 1 minuto a 13000 x
				\item Pipetear 300 $\mu$L de alcohol isoprop\'ilico a un tubo de 1.5 mL, y agregar el sobrenadante del paso anterior, y mezclar hasta observar el DNA
				\item Centrifugar por 1 minuto a 13000 x
				\item Descartar el sobrenadante
				\item Agregar 300 $\mu$L de etanol al 70 \% al contenedor del DNA previamente extra\'ido
				\item Centrifugar por 1 minuto a 13000 x
				\item Descartar el sobrenadante y secar el DNA con una corriente de aire por 5 minutos
				\item Agregar 100 $\mu$L de la soluci\'on \textit{DNA Hydratation Solution} y agitar
				\item Incubar por 5 minutos a 65 $^\circ$C para disolver el DNA
			\end{enumerate}
			
			\begin{flushright}
				Procedimiento reportado por \citeauthor{puregeneBook}
			\end{flushright}
		
		\subsection{Caracterizaci\'on}
			La determinaci\'on de la pureza del DNA extra\'ido se puede cuantificar usando espectroscop\'ia UV-Vis. Depositando una peque\~na cantidad del DNA extra\'ido en una celda de cuarzo, cuya concentraci\'on sea cercana a 20 $\mu$g/mL y midiendo el coeficiente A$_{260}$/A$_{280}$ y A$_{260}$/A$_{230}$, donde A$_\lambda$ corresponde con las absorbancias en $\lambda$ nm \cite{olson2012dna}.
			\paragraph{A$_{260}$/A$_{280}$:} Un valor cercano a 1.8 ser\'ia indicativo de un DNA puro \cite{wilfinger1997effect, green2012molecular}.
			\begin{itemize}
				\item Desviaciones positivas a este valor estar\'ian asociadas a contaminaci\'on con RNA
				\item Desviaciones negativas estar\'ian asociadas con la presencia de prote\'inas
			\end{itemize}
			\paragraph{A$_{260}$/A$_{230}$:} Los valores esperados se encuentran en el rango de 2.0 a 2.2 \cite{green2012molecular}.
			
	\section{Amplificaci\'on del gen}
		El gen que codifica para la enzima \enzima{} se encuentra en el cromosoma 12 de los seres humanos. La secuencia de nucle\'otidos se muestra a continuaci\'on y corresponde con 3859 bp \cite{NCBI}.
		\begin{Verbatim}[commandchars=\\\{\}]
GCTCTCTGCTCCTCCTGTTCGACAGTCAGCCGCATCTTCTTTTGCGTCGCCAGGTGAAGACGGGCGGAGA
GAAACCCGGGAGGCTAGGGACGGCCTGAAGGCGGCAGGGGCGGGCGCAGGCCGGATGTGTTCGCGCCGCT
GCGGGGTGGGCCCGGGCGGCCTCCGCATTGCAGGGGCGGGCGGAGGACGTGATGCGGCGCGGGCTGGGCA
TGGAGGCCTGGTGGGGGAGGGGAGGGGAGGCGTGTGTGTCGGCCGGGGCCACTAGGCGCTCACTGTTCTC
TCCCTCCGCGCAGCCGAGCCACATCGCTCAGACACCATGGGGAAGGTGAAGGTCGGAGTCAACGGGTGAG
TTCGCGGGTGGCTGGGGGGCCCTGGGCTGCGACCGCCCCCGAACCGCGTCTACGAGCCTTGCGGGCTCCG
GGTCTTTGCAGTCGTATGGGGGCAGGGTAGCTGTTCCCCGCAAGGAGAGCTCAAGGTCAGCGCTCGGACC
TGGCGGAGCCCCGCACCCAGGCTGTGGCGCCCTGTGCAGCTCCGCCCTTGCGGCGCCATCTGCCCGGAGC
CTCCTTCCCCTAGTCCCCAGAAACAGGAGGTCCCTACTCCCGCCCGAGATCCCGACCCGGACCCCTAGGT
GGGGGACGCTTTCTTTCCTTTCGCGCTCTGCGGGGTCACGTGTCGCAGAGGAGCCCCTCCCCCACGGCCT
CCGGCACCGCAGGCCCCGGGATGCTAGTGCGCAGCGGGTGCATCCCTGTCCGGATGCTGCGCCTGCGGTA
GAGCGGCCGCCATGTTGCAACCGGGAAGGAAATGAATGGGCAGCCGTTAGGAAAGCCTGCCGGTGACTAA
CCCTGCGCTCCTGCCTCGATGGGTGGAGTCGCGTGTGGCGGGGAAGTCAGGTGGAGCGAGGCTAGCTGGC
CCGATTTCTCCTCCGGGTGATGCTTTTCCTAGATTATTCTCTGGTAAATCAAAGAAGTGGGTTTATGGAG
GTCCTCTTGTGTCCCCTCCCCGCAGAGGTGTGGTGGCTGTGGCATGGTGCCAAGCCGGGAGAAGCTGAGT
CATGGGTAGTTGGAAAAGGACATTTCCACCGCAAAATGGCCCCTCTGGTGGTGGCCCCTTCCTGCAGCGC
CGGCTCACCTCACGGCCCCGCCCTTCCCCTGCCAGCCTAGCGTTGACCCGACCCCAAAGGCCAGGCTGTA
AATGTCACCGGGAGGATTGGGTGTCTGGGCGCCTCGGGGAACCTGCCCTTCTCCCCATTCCGTCTTCCGG
AAACCAGATCTCCCACCGCACCCTGGTCTGAGGTTAAATATAGCTGCTGACCTTTCTGTAGCTGGGGGCC
TGGGCTGGGGCTCTCTCCCATCCCTTCTCCCCACACACATGCACTTACCTGTGCTCCCACTCCTGATTTC
TGGAAAAGAGCTAGGAAGGACAGGCAACTTGGCAAATCAAAGCCCTGGGACTAGGGGGTTAAAATACAGC
TTCCCCTCTTCCCACCCGCCCCAGTCTCTGTCCCTTTTGTAGGAGGGACTTAGAGAAGGGGTGGGCTTGC
CCTGTCCAGTTAATTTCTGACCTTTACTCCTGCCCTTTGAGTTTGATGATGCTGAGTGTACAAGCGTTTT
CTCCCTAAAGGGTGCAGCTGAGCTAGGCAGCAGCAAGCATTCCTGGGGTGGCATAGTGGGGTGGTGAATA
CCATGTACAAAGCTTGTGCCCAGACTGTGGGTGGCAGTGCCCCACATGGCCGCTTCTCCTGGAAGGGCTT
CGTATGACTGGGGGTGTTGGGCAGCCCTGGAGCCTTCAGTTGCAGCCATGCCTTAAGCCAGGCCAGCCTG
GCAGGGAAGCTCAAGGGAGATAAAATTCAACCTCTTGGGCCCTCCTGGGGGTAAGGAGATGCTGCATTCG
CCCTCTTAATGGGGAGGTGGCCTAGGGCTGCTCACATATTCTGGAGGAGCCTCCCCTCCTCATGCCTTCT
TGCCTCTTGTCTCTTAGATTTGGTCGTATTGGGCGCCTGGTCACCAGGGCTGCTTTTAACTCTGGTAAAG
TGGATATTGTTGCCATCAATGACCCCTTCATTGACCTCAACTACATGGTGAGTGCTACATGGTGAGCCCC
AAAGCTGGTGTGGGAGGAGCCACCTGGCTGATGGGCAGCCCCTTCATACCCTCACGTATTCCCCCAGGTT
TACATGTTCCAATATGATTCCACCCATGGCAAATTCCATGGCACCGTCAAGGCTGAGAACGGGAAGCTTG
TCATCAATGGAAATCCCATCACCATCTTCCAGGAGTGAGTGGAAGACAGAATGGAAGAAATGTGCTTTGG
GGAG\textcolor{green}{GCAACTAGGATGGTGTGGCT}CCCTTGGGTATATGGTAACCTTGTGTCCCTCAATATGGTCCTGTCC
CCATCTCCCCCCCACCCCCATAGGCGAGATCCCTCCAAAATCAAGTGGGGCGATGCTGGCGCTGAGTACG
TCGTGGAGTCCACTGGCGTCTTCACCACCATGGAGAAGGCTGGGGTGAGTGCAGGAGGGCCCGCGGGAGG
GGAAGCTGACTCAGCCCTGCAAAGGCAGGACCCGGGTTCATAACTGTCTGCTTCTCTGCTGTAGGCTCAT
TTGCAGGGGGGAGCCAAAAGGGTCATCATCTCTGCCCCCTCTGCTGATGCCCCCATGTTCGTCATGGGTG
TGAACCATGAGAAGTATGACAACAGCCTCAAGATCATCAGGTGAGGAAGGCAGGGCCCGTGGAGAAGCGG
CCAGCCTGGCACCCTATGGACACGCTCCCCTGACTTGCGCCCCGCTCCCTCTTTCTTTGCAGCAATGCCT
CCTGCACCACCAACTGCTTAGCACCCCTGGCCAAGGTCATCCATGACAACTTTGGTATCGTGGAAGGACT
CATGG\textcolor{green}{TATGAGAGCTGGGGAATGGGA}\textcolor{blue}{CTGAGGCTCCCACCTTTCTC}ATCCAAGACTGGCTCCTCCCTGCC
GGGGCTGCGTGCAACCCTGGGGTTGGGGGTTCTGGGGACTGGCTTTCCCATAATTTCCTTTCAAGGTGGG
GAGGGAGGTAGAGGGGTGATGTGGGGAGTACGCTGCAGGGCCTCACTCCTTTTGCAGACCACAGTCCATG
CCATCACTGCCACCCAGAAGACTGTGGATGGCCCCTCCGGGAAACTGTGGCGTGATGGCCGCGGGGCTCT
CCAGAACATCATCCCTGCCTCTACTGGCGCTGCCAAGGCTGTGGGCAAGGTCATCCCTGAGCTGAACGGG
AAGCTCACTGGCATGGCCTTCCGTGTCCCCACTGCCAACGTGTCAGTGGTGGACCTGACCTGCCGTCTAG
AAAAACCTGCCAAATATGATGACATCAAGAAGGTGGTGAAGCAGGCGTCGGAGGGCCCCCTCAAGGGCAT
CCTGGGCTACACTGAGCACCAGGTG\textcolor{Bittersweet}{GTCTCCTCTGACTTCAACAGCG}ACACCCACTCCTCCACCTTTGAC
GCTGGGGCTGGCATTGCCCTCAACGACCACTTTGTCAAGCTCATTTCCTGGTATGTGGCTGGGGCCAGAG
ACTGGCTCTTAAAAAGTGCAGGGTCTGGCGCCCTCTGGTGGCTGGCTCAGA\textcolor{blue}{AA}\textcolor{DarkOrchid}{AAGGGCCCTGACAACTC}
\textcolor{DarkOrchid}{TT}\textcolor{red}{T}TCATCTTCTAGGTATGACAACGAAT\textcolor{Bittersweet}{TTGGCTACAGCAACAGGGTGGT}GGACCTCATGGCCCACATGG
CCTCCAAGGAGTAAGACCCCTGGACCACCAGCCCCAGCAAGAGCACAAGAGGAAGAGAGAGACCCTCACT
GCTGGGGAGTCCCTGCCACACTCAGTCCCCCACCACACTGAATCTCCCCTCCTCACAGTTGCCATGT\textcolor{red}{AGA}
\textcolor{red}{CCCCTTGAAGAGGGGAG}GGGCCTAGGGAGCCGCACCTTGTCATGTACCATCAATAAAGTACCCTGTGCTC
AACCAGTTA
		\end{Verbatim}
		\subsection{PCR}
			\subsubsection{Elecci\'on de primers}		
			Un resumen completo de los posibles primers a usar se muestra a continuaci\'on:
			\begin{table}[h]
				\centering
				\caption{Posibles primers a usar}
				\begin{tabular}{rc|cc}
					\hline
					& \textbf{Primers} & \textbf{Longitud} & \textbf{\% GC} \\
					\hline
					\textbf{Forward:} & \color{Bittersweet}GTCTCCTCTGACTTCAACAGCG & 22 & 54.5 \\
					\textbf{Reverse:} & \color{Bittersweet}ACCACCCTGTTGCTGTAGCCAA & 22 & 54.5 \footnote{A nivel comercial es posible adquirir primer de \textsc{OriGene} \url{https://www.origene.com/catalog/gene-expression/qpcr-primer-pairs/hp205798/gapdh-human-qpcr-primer-pair-nm_002046}}\\
					\hline
					\textbf{Forward:} & \color{green}GCAACTAGGATGGTGTGGCT & 20 & 55 \\
					\textbf{Reverse:} & \color{green}	TCCCATTCCCCAGCTCTCATA & 21 & 52.4 \\
					\hline
					
					\textbf{Forward:} & \color{blue}CTGAGGCTCCCACCTTTCTC & 20 & 60.0 \\
					\textbf{Reverse:} & \color{blue}	AAGAGTTGTCAGGGCCCTTTT & 21 & 47.6 \\
					\hline

					\textbf{Forward:} & \color{red}AAGGGCCCTGACAACTCTTT & 20 & 50.0 \\
					\textbf{Reverse:} & \color{red}	CTCCCCTCTTCAAGGGGTCT & 20 & 60.0 \\
					\hline
				\end{tabular}
			\end{table}
		\subsection{Purificaci\'on}
		\subsection{Secuenciaci\'on}
	\section{Selecci\'on del vector (clonaci\'on y/o expresi\'on)}
	\section{Seleccionar enzimas de restricci\'on}
		\subsection{New England Biolabs (Pl\'asmido)}
		\subsection{New England Biolabs (Gen)}
	\section{Ligar gen + pl\'asmido}
	\section{Transformarlo}
		Generar muchas r\'eplicas del pl\'asmido (procariota o eucariota)
	\section{Extraer el pl\'asmido recombinante}
	\section{Expresi\'on}
	\section{Extracci\'on y purificaci\'on}
	\section{An\'alisis estructural de la prote\'ina}
		\subsection{Estructura secundaria}
		\subsection{Modificaciones postraduccionales}
		\subsection{Estructura cuaternaria}
	\bibliography{bibliography}
	\bibliographystyle{unsrt}
%	\bibliographystyle{plainnat}
\end{document}
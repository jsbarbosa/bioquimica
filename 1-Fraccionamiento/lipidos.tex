\begin{table*}[h!]
	\small
	\caption{Resultados obtenidos para los distintos tejidos con las pruebas espec\'ificas a l\'ipidos.}
	\label{tb: lipidos}
	\begin{tabular}{p{2.5cm}|p{7cm}p{7cm}}
		\hline
		\textbf{Tejido} & \textbf{1. Prueba Salkowsky} & \textbf{2. Prueba NaOH}
		\\
		\hline
		\textbf{H\'igado de pollo} & Se observó la formación de dos capas, la superior de color verde claro y la inferior rojiza & Se observó la formación de dos capas, la superior con textura espesa. Tras calentar se tornó rojiza. 
		\\
		\hline
		\textbf{H\'igado de pollo} & No se observó la formación de las dos fases esperadas, la prueba no dió & Se observó un cambio de coloración a un color verde oscuro, el cual cambió a café al calentarse, la prueba dió.
		\\
		\hline
		
		\textbf{H\'igado de res} & Se observo la formaci\'on de dos capas, la superior verde muy tenue casi blanco, la inferior cafe claro y una pequeña capa caf\'e oscura entre las dos anteriores & Al inicio se ten\'ian tres capas de coloraci\'on vinotinto, rojiza y amarilla muy tenue, pero despu\'es de calentamiento tomo una coloraci\'on caf\'e/verde
		\\
		\hline
		
		\textbf{H\'igado de res} & La muestra presenta una coloración rojiza luego de la adición del ácido, cuando se calienta se torna de color café & Se observa en un inicio una solución de color rojo intenso, a medida que se calentó tomó un color verde
		\\
		\hline
		
		\textbf{Coraz\'on de pollo} & Se observó la formaci\'on de dos capas: de color caf\'e claro y blanco & Se observó una solución poco turbia y de color caf\'e verdoso
		\\
		\hline
		
		\textbf{Coraz\'on de pollo} & Al minuto se vio la formaci\'on de dos fases separadas por un anillo naranja. Despu\'es de 15 minutos el anillo se hab\'ia tornado café & La soluci\'on se volvi\'o verde, con el calentamiento se tornó caf\'e y finalmente volvi\'o a tomar coloraci\'on verdosa
		\\
		\hline
		
		\textbf{Cerebro de res} & No se observó precipitado & La muestra cambió a verde, al calentar se tornó café y por último volvió a verde
		\\
		\hline
		\textbf{Cerebro de res} & Tras la adición de cloroformo se formaron dos fases y la muestra se tornó color pastel. Después de adicionar anhídrido acético se obtuvo una fase. Finalmente, al agregar ácido sulfúrico se observa la formación de una tercera fase con un anillo color negro en la interfase & Al agregar NaOH la muestra cambió a un color verde. Conforme el tiempo pasó, se formaron dos fases. La superior con un color verde oscuro y la inferior con un color verde claro. Con calentamiento la fase superior cambió a color rojo mientras que la inferior cambiaba lentamente a un color amarillo. Finalmente, tras la adición de agua y agitación se formaron dos fases: una verde uniforme y otra blanca de apariencia espumosa.
		\\
		\hline
	\end{tabular}
\end{table*}

\begin{table*}[h!]
	\small
	\caption{Resultados obtenidos para los distintos tejidos con las pruebas espec\'ificas a carbohidratos.}
	\label{tb: carbohidratos}
	\begin{tabular}{p{1.5cm}|p{3.5cm}p{3.5cm}p{3.5cm}p{3.5cm}}
		\hline
		\textbf{Tejido} & \textbf{1. Prueba Lugol} & \textbf{2. Prueba ácido + lugol} & \textbf{3. Reacción de Molish} & \textbf{4. Reacción de Benedict}
		\\
		\hline
		\textbf{H\'igado de pollo} & La solución inicial de coloración amarilla no presentó ningún cambio & Tampoco se evidenció algún cambio en la solución & Se observó la formación de un anillo color violeta & La solución inicial presentaba una coloración azul. Se observó la formación de un precipitado verde
		\\
		\hline
		\textbf{H\'igado de pollo} & La coloración amarilla de la solución inicial se mantuvo, la prueba no dió & Tampoco dió & Se observó la formación del anillo morado esperado & No se observó el cambio de coloración esperado o la formación de un precipitado verde, la prueba no dió
		\\
		\hline
		
		\textbf{H\'igado de res} & La solución no cambio su coloración amarilla inicial, pero en la parte inferior del tubo se tornó amarillo mas intenso & La solución tampoco cambio su coloración amarilla inicial, y la parte inferior del tubo tambi\'en se tornó amarillo mas intenso	& Se observó la formaci\'on de un anillo color morado en la parte inferior & Primero se observó una coloraci\'on azul tenue en la parte superior y azul fuerte en la parte inferior del tubo. Despu\'es de calentamiento tomo una coloraci\'on verde tenue en la parte superior y fuerte en la inferior.
		\\
		\hline
		
		\textbf{H\'igado de res} & La solución tomó un color amarillo pálido correspondiente a la adición del reactivo, no se observa ningún cambio & No se observa un cambio aparente con respecto a su color inicial (amarillo pálido) & Se observa la formación de un anillo de color morado entre las fases presentes & La muestra es incolora, al adicionar el reactivo toma un color azul y luego de 10 minutos de calentamiento vira al verde
		\\
		\hline
		
		\textbf{Coraz\'on de pollo} & La solución arrojó una coloración cafe rojizo & Color amarillo sin algun cambio representativo & Se formaron dos fases y un anillo violeta & No hubo precipitado, ni cambio de color
		\\
		\hline
		
		\textbf{Coraz\'on de pollo} & La solucion se mantuvo amarilla, no hubo cambio en la coloracion & No se vio cambio de color	& Se formaron dos fases y en la interfase se vio un anillo morado & No reacciono. Sin cambio de color y sin precipitado rojo.
		\\
		\hline
		
		\textbf{Cerebro de res} & No dio & Inicialmente se presentó una coloración amarilla en el fondo, la cual fue desapareciendo con el tiempo. & Se observó la formación de un anillo color amarillo tenue. & La prueba arrojó coloraci\'on azul
		\\
		\hline
		\textbf{Cerebro de res} & No se observó un cambio de color & No se observó un cambio de color & Se observó la formación de dos fases. La fase que estaba en la parte más superficial se tornó de un color rojo. Tras la adición de una mayor cantidad de ácido sulfúrico se observó lentamente la formación del anillo color negro en la interfase & No se observó un cambio de color
		\\
		\hline
	\end{tabular}
\end{table*}
